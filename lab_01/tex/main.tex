\documentclass[12pt]{report}
\usepackage[utf8]{inputenc}
\usepackage[russian]{babel}
%\usepackage[14pt]{extsizes}
\usepackage{listings}
\usepackage{graphicx}
\usepackage{amsmath,amsfonts,amssymb,amsthm,mathtools} 
\usepackage{pgfplots}
\usepackage{filecontents}
\usepackage{indentfirst}
\usepackage{eucal}
\usepackage{amsmath}
\usepackage{enumitem}
\frenchspacing

\usepackage{indentfirst} % Красная строка


%\usetikzlibrary{datavisualization}
%\usetikzlibrary{datavisualization.formats.functions}

\usepackage{amsmath}




% Для листинга кода:
\lstset{ %
language=haskell,                 % выбор языка для подсветки (здесь это С)
basicstyle=\small\sffamily, % размер и начертание шрифта для подсветки кода
numbers=left,               % где поставить нумерацию строк (слева\справа)
numberstyle=\tiny,           % размер шрифта для номеров строк
stepnumber=1,                   % размер шага между двумя номерами строк
numbersep=5pt,                % как далеко отстоят номера строк от подсвечиваемого кода
showspaces=false,            % показывать или нет пробелы специальными отступами
showstringspaces=false,      % показывать или нет пробелы в строках
showtabs=false,             % показывать или нет табуляцию в строках
frame=single,              % рисовать рамку вокруг кода
tabsize=2,                 % размер табуляции по умолчанию равен 2 пробелам
captionpos=t,              % позиция заголовка вверху [t] или внизу [b] 
breaklines=true,           % автоматически переносить строки (да\нет)
breakatwhitespace=false, % переносить строки только если есть пробел
escapeinside={\#*}{*)}   % если нужно добавить комментарии в коде
}

\usepackage[left=2cm,right=2cm, top=2cm,bottom=2cm,bindingoffset=0cm]{geometry}
% Для измененных титулов глав:
\usepackage{titlesec, blindtext, color} % подключаем нужные пакеты
\definecolor{gray75}{gray}{0.75} % определяем цвет
\newcommand{\hsp}{\hspace{20pt}} % длина линии в 20pt
% titleformat определяет стиль
\titleformat{\chapter}[hang]{\Huge\bfseries}{\thechapter\hsp\textcolor{gray75}{|}\hsp}{0pt}{\Huge\bfseries}


% plot
\usepackage{pgfplots}
\usepackage{filecontents}
\usetikzlibrary{datavisualization}
\usetikzlibrary{datavisualization.formats.functions}
\RequirePackage[
  style=gost-numeric,
  language=auto,
  autolang=other,
  sorting=none,
]{biblatex}

\addbibresource{bib.bib}
\begin{document}
%\def\chaptername{} % убирает "Глава"
\thispagestyle{empty}
\begin{titlepage}
	\noindent \begin{minipage}{0.15\textwidth}
	\includegraphics[width=\linewidth]{b_logo}
	\end{minipage}
	\noindent\begin{minipage}{0.9\textwidth}\centering
		\textbf{Министерство науки и высшего образования Российской Федерации}\\
		\textbf{Федеральное государственное бюджетное образовательное учреждение высшего образования}\\
		\textbf{~~~«Московский государственный технический университет имени Н.Э.~Баумана}\\
		\textbf{(национальный исследовательский университет)»}\\
		\textbf{(МГТУ им. Н.Э.~Баумана)}
	\end{minipage}
	
	\noindent\rule{18cm}{3pt}
	\newline\newline
	\noindent ФАКУЛЬТЕТ $\underline{\text{«Информатика и системы управления»}}$ \newline\newline
	\noindent КАФЕДРА $\underline{\text{«Программное обеспечение ЭВМ и информационные технологии»}}$\newline\newline\newline\newline\newline
	
	
	\begin{center}
		\noindent\begin{minipage}{1.3\textwidth}\centering
			\Large\textbf{  Отчёт по лабораторной работе №1 по дисциплине}\newline
			\textbf{ "Основы искусственного интеллекта"}\newline\newline
		\end{minipage}
	\end{center}
	
	\noindent\textbf{Тема} $\underline{\text{Опрос эксперта}}$\newline\newline
	\noindent\textbf{Студент} $\underline{\text{Варламова Е. А.}}$\newline\newline
	\noindent\textbf{Группа} $\underline{\text{ИУ7-13М}}$\newline\newline
	\noindent\textbf{Оценка (баллы)} $\underline{\text{~~~~~~~~~~~~~~~~~~~~~~~~~~~}}$\newline\newline
	\noindent\textbf{Преподаватели} $\underline{\text{Строганов Ю.В.}}$\newline\newline\newline
	
	\begin{center}
		\vfill
		Москва~---~\the\year
		~г.
	\end{center}
\end{titlepage}
\large
\setcounter{page}{2}
\def\contentsname{СОДЕРЖАНИЕ}
\renewcommand{\contentsname}{СОДЕРЖАНИЕ}
\tableofcontents
\renewcommand\labelitemi{---}
\newpage
\chapter*{Введение}
\addcontentsline{toc}{chapter}{ВВЕДЕНИЕ}

Опрос эксперта может быть полезен для получения квалифицированной оценки или мнения по определенному вопросу или проблеме. 
Эксперт обладает специальными знаниями и опытом в определенной области, поэтому его мнение может быть ценным для принятия решений, проведения исследований, разработки ПО. 
Опрос эксперта может использоваться в научных исследованиях, бизнес-анализе, политических дискуссиях и других областях, где важно иметь качественные и авторитетные точки зрения.

Целью данной лабораторной работы является создание системы автоматизации на тему <<Требования в детских медицинских, социально-медицинских, образовательных учреждениях к сотрудникам + штрафы за нарушение>> с помощью опроса эксперта.

Для этого необходимо решить следующие задачи:
\begin{itemize}
    \item ознакомиться с тематикой;
    \item предложить тему автоматизации и уточнить её с помощьью эксперта;
    \item описать общую концепции разработки;
    \item описать виды пользователей, варианты использования приложения;
    \item описать наиболее распространённый сценарий;
    \item разработать схему ПО;
    \item реализовать ПО;
    \item протестировать ПО, привести примеры работы.
\end{itemize}

\chapter{Аналитическая часть}

\section{Ознакомление с тематикой}
В результате общения с экспертом было выяснено, что основным требованием к сотрудникам в детских медицинских, социально-медицинских, образовательных учреждениях является наличие медицинской книжки и своевременное прохождение медицинского осмотра в соответствии с приказом 29н \cite{29n}.

При не своевременном прохождении медосмотра работником предусмотрены штрафы, накладываемые на работодателя за допуск такого работника к исполнению трудовых обязанностей, в соответсвии с КоАП РФ Статья 5.27.1.

\section{Тема и общая концепция разработки} 

Тема -- автоматизация процесса периодической выдачи электронных направлений сотрудникам в детских медицинских, социально-медицинских, образовательных учреждениях на прохождение медосмотра и получения от сотрудника медицинской книжки в электронном виде.

Предлагается создать приложение, позволяющее автоматизировать процесс периодической (в частности, ежегодной) выдачи направлений на медосмотр в электронном виде работникам от работодателя, а также загрузки работником полученной (обновленной) книжки в систему внутреннего документооборота работодателя. 

В соответствии с требованиями трудового кодекса все документы в электронном виде, участвующие во взаимодействии работника и работодателя, должны быть подписаны с помощью электронной подписи.
Поэтому в приложении должна быть предусмотрена возможность подписания работодателем направления на прохождение медосмотра и проверка подписи (выданной медицинской организацией) медицинской книжки. 

Также необходимо организовать напоминание сотрудникам об обновлении книжки и прохождении медосмотра с ссылкой на КоАП РФ (о штрафах за нарушение).

\newpage

\section{Виды пользователей}
В приложении предлагается ввести 2 вида пользователей, имеющих возоможность создать аккаунт в системе и зайти в личный кабинет:
\begin{itemize}
    \item работодатель -- в личном кабинете пользователи этой роли должны иметь возможность: подписать направление на прохождение медосмотра сотрудником, отобразить подписанное направление в личный кабинет сотрудника, просмотреть состояние медицинских книжек всех сотрудников (являются ли книжки действующими, прошли ли проверку подлинности подписи), также просмотреть направления каждого сотрудника;
    \item сотрудник -- в личном кабинете пользователи этой роли должны иметь возможность получить информацию о сроке окончания действия медицинской книжки (с указанием на размер штрафа в случае несвоевременного обновления),  доступных направлениях на медосмотр, также иметь возможность загрузить в систему медицинскую книжку, подписанную медицинской организацией, выдавшей эту книжку, и увидеть результат проверки подлинности подписи.
\end{itemize}

Use-case диаграмма представлена на рисунке \ref{fig:uc}. 
ER-диаграмма диаграмма представлена на рисунке \ref{fig:er}.

\begin{figure}[h!]
  \centering
  \includegraphics[width = \linewidth]{use-case.pdf}
  \caption{Use-case диаграмма}
  \label{fig:uc}
\end{figure}
\newpage
\begin{figure}[h!]
  \centering
  \includegraphics[width = \linewidth]{er.pdf}
  \caption{ER-диаграмма}
  \label{fig:er}
\end{figure}

\section{Описание наиболее распространённого сценария}
\begin{enumerate}
    \item Работодатель через свой личный кабинет загружает направления на медосмотр сотрудникам, срок действия медицинских книжек которых истекает в ближайшее время. 
    \item После этого каждый сотрудник может увидеть в своём личном кабинете направление на медосмотр в электронном виде. Кроме того, в личном кабинете отражена информация о сроке окончания действия текущей книжки, а также информация о штрафах, предусмотренных в случае предоставления фальшивой медицинской книжки/за её отсутствие.
    \item Сотрудник приходит с электронным направлением в организацию, проводящую медосмотр. Организация в проверяет подлинность направления (предполагается, что медицинская органиазция, проводящая медосмотр, и работодатель заранее обменялись публичными ключами подписей).
    \item Сотрудник проходит медосмотр и ждет от организации, проводящей медосмотр, медицинскую книжку в электронном виде (подписанную организацией).
    \item После получения медицинской книжки в электронном виде от организации сотрудник загружает ее в свой личный кабинет, где автоматически проверяется подлинность книжки и выдается результат проверки.
    \item Работодатель проверяет, что все медицинские книжки, на обновление которых были выданы направления, были обновлены в срок и что все проверки подлинности книжек были пройдены.
\end{enumerate}



\section*{Вывод}
\addcontentsline{toc}{section}{Вывод}
В данном разделе была приведена тема автоматизации, составленная по результатам опроса эксперта, приведена общая концепция разрбаотки приложения, описаны виды пользователей приложения, а также описан наиболее распрстранённый сценарий.


\chapter{Конструкторская часть}

\section{Используемые структуры данных}
Предлагается создать 3 структуры данных, описывающих 2 вида пользователей системы:
\begin{itemize}
    \item структура данных Account, которая содержит в себе общие для любого аккаунта поля: логин и пароль;
    \item структура данных HRAccount, которая описывает роль работодателя в системе; структура содержит ссылку на Account, а также поле private\_key, являющееся приватным ключом каждого представителя работодателя; поле private\_key используется для создания электронной подписи направлений на медосмотр;
    \item структура данных EmploeeAccount, которая описывает роль работника в системе; структура содержит ссылку на Account, а также поля: 
    \begin{itemize}
        \item медицинская книжка;
        \item сертификат медицинской книжки;
        \item срок действия медицинской книжки;
        \item направление на медицинский осмотр;
        \item сертификат направления на медицинский осмотр;
        \item срок действия направления на медицинский осмотр.
    \end{itemize}
\end{itemize}

\section{Компонентная диаграмма разрабатывамого ПО}

В программном обеспечении, реализующем автоматизацию процесса периодической выдачи электронных направлений сотрудникам, предлагается выделить четыре следующих компонента. 

\begin{itemize}  \renewcommand\labelitemi{---} 
    \item Компонент \textbf{<<Отображение данных>>}. Данный компонент отвечает за представление всех результатов ПО, а также предоставление возможности задать все входные данные. 
    \item Компонент \textbf{<<Логика>>}. Данный компонент отвечает за реализацию логики работы приложения, то есть реализует логику регистрации в приложении, входа в приложение, подписания направлений, проверки достоверности подписи на медицинской книжке, получения данных из/загрузку данных в постоянное хранилище.
    \item Компонент \textbf{<<Доступ к данным>>}. Данный компонент отвечает за предоставление интерфейса для получения данных из базы данных; этот компонент используется компонентом \textbf{<<Логика>>}.
    \item Компонент \textbf{<<База данных>>}. Данный компонент отвечает непосредственно за хранение данных.
\end{itemize}

При этом компонент \textbf{<<База данных>>} является внешним по отношению к разрабатываемому ПО. Связь компонентов представлена на рисунке \ref{fig:structure}.

\begin{figure}[h!]
  \centering
  \includegraphics[width = \linewidth]{structure.pdf}
  \caption{Связь компонентов приложения}
  \label{fig:structure}
\end{figure}


\subsection{Описание компонента <<Отображение данных>>}
Интерфейс приложения предлагается реализовать в консольном варианте. Так, компонет <<Отображение данных>> состоит из одного класса: <<ConsoleInterface>>, который имеет метод <<Run>>, запускающий приложение. На рисунке \ref{fig:uml_ui} представлена диаграмма классов компонента \textbf{<<Отображение данных>>}.

\begin{figure}[h!]
	\centering
	\includegraphics[scale = 0.5]{interface.pdf}
	\caption{Диаграмма классов компонента \textbf{Отображение данных}}
	\label{fig:uml_ui}
\end{figure}

\subsection{Описание компонента <<Логика>>}
Компонент \textbf{<<Логика>>} включает в себя основную логику работы приложения. На рисунке \ref{fig:uml_bl} представлена концептуальная диаграмма классов компонента \textbf{<<Логика>>}.

\begin{figure}[h!]
	\centering
	\includegraphics[width = \linewidth]{BL.pdf}
	\caption{Диаграмма классов компонента \textbf{<<Логика>>}}
	\label{fig:uml_bl}
\end{figure}

Опишем назначение классов на диаграмме:

\begin{itemize}  \renewcommand\labelitemi{---} 
    \item класс BLFacade является интерфейсом компонента логики по отношению к компоненту интерфейса;
    \item класс Signature представляет предоставляет методы для создания ключей, создания подписи и проверки подписи;
    \item классы Account, HRAccount и EmploeeAccount представляют собой классы, описывающие роли пользователей системы. 
\end{itemize}

\subsection{Описание компонента <<Доступ к данным>>}

На рисунке \ref{fig:uml_da} представлена концептуальная диаграмма классов компонента \textbf{<<Доступ к данным>>}.

\begin{figure}[h!]
	\centering
	\includegraphics[scale=0.8]{DA.pdf}
	\caption{Диаграмма классов компонента \textbf{<<Доступ к данным>>}}
	\label{fig:uml_da}
\end{figure}

Опишем назначение классов на диаграмме:

\begin{itemize}  \renewcommand\labelitemi{---} 
    \item класс EmploeeAccountRepository реализует архитектурный паттерн <<Репозиторий>> для доступа к таблице в базе данных, хранящей информацию о всех работниках;
    \item класс HRAccountRepository реализует архитектурный паттерн <<Репозиторий>> для доступа к таблице в базе данных, хранящей информацию о всех представителях работодателя;
    \item класс OrganizationRepository реализует архитектурный паттерн <<Репозиторий>> для доступа к таблице в базе данных, хранящей публичные ключи организаций, выдающих медицинские книжки.
\end{itemize}


\section*{Вывод}
\addcontentsline{toc}{section}{Вывод}
В данном разделе были описаны струкутры данных разрабатывамого ПО, а также были выделены компоненты ПО, реализующего автоматизацию процесса периодической выдачи электронных направлений сотрудникам, описаны ответственности компонентов и указаны связи между ними в виде компонентной диаграммы. Для каждого компонента, не являющегося внешним по отношению к разрабтываемому ПО, была приведена концептуальная диаграмма классов компонента. 

\chapter{Технологическая часть}

\section{Выбор средств разработки}
В качестве языка программирования был использован язык Python, поскольку этот язык кроссплатформенный и для него разработано огромное количество библиотек и модулей, решающих разнообразные задачи. 

В частности, имеется библиотека, предоставляющая реализацию алгоритма шифрования RSA \cite{bib:rsa-lib}.  

Для хранения данных и работы с ними была выбрана СУБД Sqlite \cite{bib:11}, так как она реализует быстрый, надежный и полнофункциональный доступ к базе данных SQL. 

\section{Примеры работы}
\subsection{Регистрация в приложении}
На рисунке \ref{fig:reg-worker} представлен пример регистрации в приложении в роли работника.
\begin{figure}[h!]
	\centering
	\includegraphics[scale=0.8]{reg-worker.png}
	\caption{Регистрация в приложении в роли работника}
	\label{fig:reg-worker}
\end{figure}

На рисунке \ref{fig:reg-hr} представлен пример регистрации в приложении в роли работодателя.
\begin{figure}[h!]
	\centering
	\includegraphics[width = \linewidth]{reg-hr.png}
	\caption{Регистрация в приложении в роли работодателя}
	\label{fig:reg-hr}
\end{figure}

\subsection{Вход в аккаунт}

На рисунке \ref{fig:log-worker} представлен пример входа в приложение пользователя, имеющего роль работника (аккаунт, созданный в предыдущем примере).
\begin{figure}[h!]
	\centering
	\includegraphics[scale=0.8]{log-worker.png}
	\caption{Вход в приложение в роли работника}
	\label{fig:log-worker}
\end{figure}

На рисунке \ref{fig:log-hr} представлен пример входа в приложение пользователя, имеющего роль работодателя (аккаунт, созданный в предыдущем примере).
\begin{figure}[h!]
	\centering
	\includegraphics[scale=0.8]{log-hr.png}
	\caption{Вход в приложение в роли работодателя}
	\label{fig:log-hr}
\end{figure}

\subsection{Выдача направлений}

На рисунке \ref{fig:dir-hr} представлен пример создания направления представителем работодателя для одного из работников и просмотр медицинских книжек и направлений всех работников. 
Кроме того, предусмотрен вывод сообщения о штрафах, накладываемых на организацию в случае предоставления рабочих мест работникам, не имеющим медицинскую книжку. 

\begin{figure}[h!]
	\centering
	\includegraphics[width = \linewidth]{hr-direction.png}
	\caption{Создание направления представителем работодателя}
	\label{fig:dir-hr}
\end{figure}

Выписанное направление появляется в личном кабинете сотрудника, как показано на рисунке \ref{dir-worker}. 
Кроме того, если медицинская книжка просрочена, сотруднику выдаётся сооющение-напоминание о штрафах, предусмотренных за отсутствие медицинской книжки.

\begin{figure}[h!]
	\centering
	\includegraphics[width = \linewidth]{worker-direction.png}
	\caption{Направление в личном кабинете сотрудника}
	\label{dir-worker}
\end{figure}


Кроме того, сотрудник может выгрузить направление для предоставления его в медицинскую организацию, проводящую осмотры, как показано на рисунке \ref{dir-worker-load}. 


\begin{figure}[h!]
	\centering
	\includegraphics[width = \linewidth]{worker-load-direction.png}
	\caption{Загрузка направления из личного кабинета сотрудника}
	\label{dir-worker-load}
\end{figure}


\subsection{Загрузка медицинской книжки}

Работник после прохождения медосмотра должен загрузить медицинскую книжку в систему. 
После загрузки книжка проверяется на достоверность (проверяется, что электронная подпись выдана одной из известных работодателю организаций, то есть чьи публичные ключи хранятся у работодателя).
В случае успешной проверки медицинская книжка загружается в личный кабинет работодателя, как показано на рисунке \ref{fig:medbook-success}.

\begin{figure}[h!]
	\centering
	\includegraphics[scale=0.8]{medbook-success.png}
	\caption{Загрузка медицинской книжки: успешная проверка}
	\label{fig:medbook-success}
\end{figure}


В случае неуспешной проверки книжка не загружается в личный кабинет сотрудника и выдаётся сообщение об ошибке, как показано на рисунке \ref{fig:medbook-failure}.

\begin{figure}[h!]
	\centering
	\includegraphics[width = \linewidth]{medbook-failure.png}
	\caption{Загрузка медицинской книжки: неудачная проверка}
	\label{fig:medbook-failure}
\end{figure}
\newpage
\section*{Вывод}
\addcontentsline{toc}{section}{Вывод}
В данном разделе был обоснован выбор программных средств реализации ПО, а также были приведены примеры работы ПО.

\chapter*{ЗАКЛЮЧЕНИЕ}
\addcontentsline{toc}{chapter}{ЗАКЛЮЧЕНИЕ}
Целью данной лабораторной работы являлось создание системы автоматизации на тему <<Требования в детских медицинских, социально-медицинских, образовательных учреждениях к сотрудникам + штрафы за нарушение>> с помощью опроса эксперта.

Цель работы была достигнута

Для этого было необходимо решить следующие задачи:
\begin{itemize}
    \item ознакомиться с тематикой;
    \item предложить тему автоматизации и уточнить её с помощьью эксперта;
    \item описать общую концепции разработки;
    \item описать виды пользователей, варианты использования приложения;
    \item описать наиболее распространённый сценарий;
    \item разработать схему ПО;
    \item реализовать ПО;
    \item протестировать ПО, привести примеры работы.
\end{itemize}
\clearpage

\printbibliography[title={СПИСОК ИСПОЛЬЗОВАННЫХ\\ ИСТОЧНИКОВ}]
\addcontentsline{toc}{chapter}{СПИСОК ИСПОЛЬЗОВАННЫХ ИСТОЧНИКОВ}

\end{document}