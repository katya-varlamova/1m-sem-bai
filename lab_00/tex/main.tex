\documentclass[12pt]{report}
\usepackage[utf8]{inputenc}
\usepackage[russian]{babel}
%\usepackage[14pt]{extsizes}
\usepackage{listings}
\usepackage{graphicx}
\usepackage{amsmath,amsfonts,amssymb,amsthm,mathtools} 
\usepackage{pgfplots}
\usepackage{filecontents}
\usepackage{indentfirst}
\usepackage{eucal}
\usepackage{amsmath}
\usepackage{enumitem}
\frenchspacing

\usepackage{indentfirst} % Красная строка


%\usetikzlibrary{datavisualization}
%\usetikzlibrary{datavisualization.formats.functions}

\usepackage{amsmath}




% Для листинга кода:
\lstset{ %
language=haskell,                 % выбор языка для подсветки (здесь это С)
basicstyle=\small\sffamily, % размер и начертание шрифта для подсветки кода
numbers=left,               % где поставить нумерацию строк (слева\справа)
numberstyle=\tiny,           % размер шрифта для номеров строк
stepnumber=1,                   % размер шага между двумя номерами строк
numbersep=5pt,                % как далеко отстоят номера строк от подсвечиваемого кода
showspaces=false,            % показывать или нет пробелы специальными отступами
showstringspaces=false,      % показывать или нет пробелы в строках
showtabs=false,             % показывать или нет табуляцию в строках
frame=single,              % рисовать рамку вокруг кода
tabsize=2,                 % размер табуляции по умолчанию равен 2 пробелам
captionpos=t,              % позиция заголовка вверху [t] или внизу [b] 
breaklines=true,           % автоматически переносить строки (да\нет)
breakatwhitespace=false, % переносить строки только если есть пробел
escapeinside={\#*}{*)}   % если нужно добавить комментарии в коде
}

\usepackage[left=2cm,right=2cm, top=2cm,bottom=2cm,bindingoffset=0cm]{geometry}
% Для измененных титулов глав:
\usepackage{titlesec, blindtext, color} % подключаем нужные пакеты
\definecolor{gray75}{gray}{0.75} % определяем цвет
\newcommand{\hsp}{\hspace{20pt}} % длина линии в 20pt
% titleformat определяет стиль
\titleformat{\chapter}[hang]{\Huge\bfseries}{\thechapter\hsp\textcolor{gray75}{|}\hsp}{0pt}{\Huge\bfseries}


% plot
\usepackage{pgfplots}
\usepackage{filecontents}
\usetikzlibrary{datavisualization}
\usetikzlibrary{datavisualization.formats.functions}
\RequirePackage[
  style=gost-numeric,
  language=auto,
  autolang=other,
  sorting=none,
]{biblatex}

\addbibresource{bib.bib}
\begin{document}
%\def\chaptername{} % убирает "Глава"
\thispagestyle{empty}
\begin{titlepage}
	\noindent \begin{minipage}{0.15\textwidth}
	\includegraphics[width=\linewidth]{b_logo}
	\end{minipage}
	\noindent\begin{minipage}{0.9\textwidth}\centering
		\textbf{Министерство науки и высшего образования Российской Федерации}\\
		\textbf{Федеральное государственное бюджетное образовательное учреждение высшего образования}\\
		\textbf{~~~«Московский государственный технический университет имени Н.Э.~Баумана}\\
		\textbf{(национальный исследовательский университет)»}\\
		\textbf{(МГТУ им. Н.Э.~Баумана)}
	\end{minipage}
	
	\noindent\rule{18cm}{3pt}
	\newline\newline
	\noindent ФАКУЛЬТЕТ $\underline{\text{«Информатика и системы управления»}}$ \newline\newline
	\noindent КАФЕДРА $\underline{\text{«Программное обеспечение ЭВМ и информационные технологии»}}$\newline\newline\newline\newline\newline
	
	
	\begin{center}
		\noindent\begin{minipage}{1.3\textwidth}\centering
			\Large\textbf{  Отчёт по лабораторной работе №0 по дисциплине}\newline
			\textbf{ "Основы искусственного интеллекта"}\newline\newline
		\end{minipage}
	\end{center}
	
	\noindent\textbf{Тема} $\underline{\text{Знакомство с системой выдачи-приёма заданий}}$\newline\newline
	\noindent\textbf{Студент} $\underline{\text{Варламова Е. А.}}$\newline\newline
	\noindent\textbf{Группа} $\underline{\text{ИУ7-13М}}$\newline\newline
	\noindent\textbf{Оценка (баллы)} $\underline{\text{~~~~~~~~~~~~~~~~~~~~~~~~~~~}}$\newline\newline
	\noindent\textbf{Преподаватели} $\underline{\text{Строганов Ю.В.}}$\newline\newline\newline
	
	\begin{center}
		\vfill
		Москва~---~\the\year
		~г.
	\end{center}
\end{titlepage}
\large
\setcounter{page}{2}
\def\contentsname{СОДЕРЖАНИЕ}
\renewcommand{\contentsname}{СОДЕРЖАНИЕ}
\tableofcontents
\renewcommand\labelitemi{---}
\newpage
\chapter*{Введение}
\addcontentsline{toc}{chapter}{ВВЕДЕНИЕ}

Целью данной работы является оформление отчета, содержащего ER-диаграмму, IDEF0-диаграмму DFD-диаграмму, BPMN2.0 диаграмму, диаграмму последовательностей и диаграмму прецедентов для произвольной (не обязательно единой) предметной области, а также знакомство с системой выдачи-приёма заданий. 
Для этого надо решить следующие задачи:
\begin{itemize}
    \item проанализировать назначение каждой из диаграмм.
    \item изучить стандарты оформления диаграмм;
    \item изучить требования к оформлению отчётов;
    \item разработать диаграммы в соответствии со стандартами;
    \item обосновать выбор средств для построения диаграмм.
\end{itemize}

\chapter{Аналитическая часть}

\section{Назначение ER-диаграммы}
Наиболее известным представителем класса семантических моделей предметной области является модель «сущность-связь» или ER-модель. 
ER-модель представляет собой формальную конструкцию, которая сама по себе не предписывает никаких графических средств её визуализации.
В качестве стандартной графической нотации, с помощью которой можно визуализировать ERM, была предложена диаграмма сущность-связь (entity-relationship diagram,
ERD). 
На практике понятия ER-модель и ER-диаграмма часто не различают, хотя для визуализации ER-моделей предложены и другие графические нотации. 

\section{Назначение IDEF0-диаграммы}
Методология функционального моделирования и графическая нотация, предназначенная для формализации и описания бизнес-процессов.
Отличительной особенностью IDEF0 является её акцент на соподчинённость объектов. 
В IDEF0 рассматриваются логические отношения между работами, а не их временная последовательность. 

\section{Назначение DFD-диаграммы}
DFD-диаграмма (диаграмма потоков данных) -- это графическое представление потоков данных в системе, процессов, хранилищ данных и внешних сущностей. 
Она используется для моделирования бизнес-процессов и информационных систем, а также для документирования требований к системам. 
DFD-диаграммы помогают понять, как данные перемещаются в системе, какие процессы их обрабатывают и какие хранилища используются. 
DFD-диаграммы могут быть использованы на разных этапах жизненного цикла системы -- от анализа требований до тестирования и сопровождения.

\section{Назначение BPMN2.0 диаграммы}
BPMN (Business Process Modeling Notation) -- это графическая нотация для моделирования бизнес процессов. 
Основная цель, которую ставили разработчики спецификации BPMN — создание стандартной нотации понятной широкому кругу бизнес пользователей: бизнес-аналитикам, создающим и улучшающим процессы компании, техническим разработчикам, ответственным за реализацию процессов, менеджерам, следящим за работой предприятия и управляющих им.
Стандарт BPMN является доступным в интернет документом. Он детально описывает особенности реализации нотации.

\section{Назначение диаграммы последовательностей}
Диаграмма последовательностей -- это графическое представление последовательности действий в системе или приложении, отображающее взаимодействие между объектами.
Основными элементами диаграммы последовательности являются обозначения объектов, вертикальные <<линии жизни>>, отображающие течение времени, прямоугольники, отражающие деятельность объекта или исполнение им определенной функции, и стрелки, показывающие обмен сигналами или сообщениями между объектами.

\section{Назначение диаграммы прецедентов}
Диаграмма прецедентов -- диаграмма, отражающая отношения между акторами и прецедентами. 
Прецедент -- возможность моделируемой системы, благодаря которой пользователь может получить конкретный, измеримый и нужный ему результат. 
Прецедент соответствует отдельному сервису системы, определяет один из вариантов её использования и описывает типичный способ взаимодействия пользователя с системой. Варианты использования обычно применяются для спецификации внешних требований к системе.

\section{Стандарты оформления диаграмм}
Диаграммы должны соответствовать стандартам UML \cite{bib:1} и BPMN \cite{bib:2}.

\section*{Вывод}
\addcontentsline{toc}{section}{Вывод}
	В данном разделе были проанализированы назначения каждой из диаграмм и изучены стандарты оформления диаграмм.
\clearpage

\chapter{Конструкторская часть}
\section{ER-диаграмма}
Предметная область: библиотечная система, которая позволит читателям получать информацию о доступных книгах в разных библиотеках, библиотекарям -- выдавать и принимать книги, а администраторам библиотечной системы -- редактировать информацию о книгах и библиотеках. 
На рисунке \ref{fig:er} представлена ER-диаграмма библиотечной системы. 
\begin{figure}[h]
  \centering
  \includegraphics[scale = 0.6]{er.pdf}
  \caption{ER-диаграмма библиотечной системы}
  \label{fig:er}
\end{figure}

\section{IDEF0-диаграмма}
Предметная область: для процесса, путь к исполняемому файлу которого подаётся на вход, и доступным на ядре процессора частотам метод динамически настраивает частоту ядра процессора во время исполнения процесса, используя метрику MPI (Misses Per Instructions).

Формализация задачи в виде IDEF0-диаграммы изображена на рисунках \ref{fig:idef0-0}-\ref{fig:dec}.

\begin{figure}[h]
  \centering
  \includegraphics[scale = 0.8]{idef0-0.pdf}
  \caption{Формализация задачи: верхний уровень}
  \label{fig:idef0-0}
\end{figure}

\begin{figure}[h]
  \centering
  \includegraphics[scale = 0.6]{idef0-1.pdf}
  \caption{Формализация задачи: второй уровень}
  \label{fig:dec}
\end{figure}





\section{DFD-диаграмма}
Предметная область: при описании устройства и работы электрического прибора (фена) была выделена сущность -- мотор. 
На рисунке \ref{fig:dfd} представлена диаграмма состояния <<выключен>> мотора фена.


\begin{figure}[h]
  \centering
  \includegraphics[scale = 0.8]{dfd.pdf}
  \caption{DFD-диаграмма}
  \label{fig:dfd}
\end{figure}


\section{Диаграмма прецедентов}
Предметная область: библиотечная система, которая позволит читателям получать информацию о доступных книгах в разных библиотеках, библиотекарям -- выдавать и принимать книги, а администраторам библиотечной системы -- редактировать информацию о книгах и библиотеках. 
На рисунке \ref{fig:uc} представлена диаграмма прецедентов библиотечной системы. 
\newpage
\begin{figure}[h]
  \centering
  \includegraphics[scale=0.7]{uc.pdf}
  \caption{Диаграмма прецедентов библиотечной системы}
  \label{fig:uc}
\end{figure}

\section{Диаграмма последовательностей}

Предметная область: описание системного вызова select.
На рисунке \ref{fig:mc} представлена диаграмма последовательностей.

\begin{figure}[h]
  \centering
  \includegraphics[scale = 0.9]{ds.pdf}
  \caption{Диаграмма последовательностей select}
  \label{fig:mc}
\end{figure}

\newpage
\section{BPMN2.0 диаграмма}
Предметная область: проверка корректности оцифровки доверенностей в банке.
МГ -- это мобильная группа. 
АС -- это автоматизированная система.

\begin{figure}[h]
  \centering
  \includegraphics[width= \linewidth]{bpmn.pdf}
  \caption{Диаграмма BPMN2.0}
  \label{fig:mc}
\end{figure}

\section*{Вывод}
\addcontentsline{toc}{section}{Вывод}
В данном разделе были разработаны диаграммы в соответствии со стандартами.

\chapter{Технологическая часть}


\section{Средства построения диаграмм}
Drawio и StarUML являются универсальными инструментами для создания различных типов диаграмм, включая ER-диаграммы, IDEF0-диаграммы, DFD-диаграммы, BPMN2.0 диаграммы, диаграммы последовательностей и диаграммы прецедентов.

Drawio и StarUML предоставляют функции для создания и редактирования диаграмм и широкий набор символов и обозначений:
\begin{itemize}
    \item для ER-диаграмм: для описания сущностей, атрибутов и связей между ними
    \item для DFD-диаграмм: для описания процессов, хранилищ данных и потоков данных между ними;
    \item для BPMN2.0 диаграмм: для описания элементов бизнес-процесса, а также функций для создания и редактирования диаграмм
    \item для диаграмм последовательностей: для описания объектов и сообщений между ними;
    \item для диаграммы прецедентов: для описания актеров, прецедентов и их связей.
\end{itemize}

\section*{Вывод}
\addcontentsline{toc}{section}{Вывод}
В данном разделе были обоснованы средства построения диаграмм.

\chapter*{ЗАКЛЮЧЕНИЕ}
\addcontentsline{toc}{chapter}{ЗАКЛЮЧЕНИЕ}
Целью данной работы являлось оформление отчета, содержащего ER-диаграмму, IDEF0-диаграмму DFD-диаграмму, BPMN2.0 диаграмму, диаграмму последовательностей и диаграмму прецедентов для произвольной (не обязательно единой) предметной области, а также знакомство с системой выдачи-приёма заданий. 

Цель работы достигнута. Для этого были решены следующие задачи:
\begin{itemize}
    \item проанализировано назначение каждой из диаграмм.
    \item были изучены стандарты оформления диаграмм;
    \item были изучены требования к оформлению отчётов;
    \item разработаны диаграммы в соответствии со стандартами;
    \item обоснован выбор средств для построения диаграмм.
\end{itemize}

\printbibliography[title={СПИСОК ИСПОЛЬЗОВАННЫХ\\ ИСТОЧНИКОВ}]
\addcontentsline{toc}{chapter}{СПИСОК ИСПОЛЬЗОВАННЫХ ИСТОЧНИКОВ}

\end{document}